\documentclass[11pt,a4paper]{article}
\usepackage[utf8]{inputenc}
\usepackage[T1]{fontenc}
\usepackage{lmodern}
\usepackage{hyperref}
\usepackage{geometry}
\geometry{
a4paper,
total={170mm,257mm},
left=20mm,
top=20mm,
}

\begin{document}

\begin{center}
{\LARGE\textbf{Jinglei Cheng}} \\
\medskip
Department of Computer Science \textbar{} University of Pittsburgh \textbar{} Pittsburgh, PA, 15213-4034, USA\\
Tel: (+1) 323-393-5936 \textbar{} Email: \href{mailto:jic373@pitt.edu}{jic373@pitt.edu}
\end{center}

\section*{EDUCATION}

\begin{itemize}

\item 
\textbf{University of Pittsburgh}, Pittsburgh, USA \hfill Nov 2024 - Now

\textbf{Postdoctoral Associate}

Department of Computer Science

\item 
\textbf{Purdue University}, West Lafayette, USA \hfill Aug 2022 -- Aug 2024

\textbf{Ph.D. of Computer Science}

Department of Computer Science

\item
\textbf{University of Southern California}, Los Angeles, USA \hfill Aug 2018 -- Aug 2022

\textbf{Master Electrical Engineering, Ph.D. Electrical Engineering (transferred to Purdue University)}

Ming Hsieh Department of Electrical and Computer Engineering    

\item 
\textbf{Tsinghua University}, Beijing, China \hfill Sept 2014 -- June 2018

\textbf{B. E. in Microelectronic Science and Engineering at Tsinghua University}, China

Department of Microelectronics and Nanoelectronics    \\

\textbf{Additional Notes:}

\item \textbf{Summer Break}, Jingmen, China \hfill July 2018

I was enjoying my summer break and preparing for my upcoming travel to the USA as Ph.D. student.

\item \textbf{Artephi Computing Inc.}, West Lafayette, USA \hfill Sept 2024 -- Oct 2024

I was shortly employed by the startup company Artephi Computing Inc., where I helped develop optimization software, before being recruited by the University of Pittsburgh as a postdoctoral associate.

\end{itemize}






\section*{RESEARCH INTERESTS}

My research interests lie in the following areas: 
\begin{itemize}
    \item Interaction Between Quantum Computing and Physics: Exploring how fundamental principles of physics inform and enhance quantum computing methodologies and how quantum computing can produce new phenomenons.
    \item Quantum Neural Tangent Kernels (QNTK), Variational Quantum Algorithms (VQA), and Distributed Quantum Computing (DQC): Investigating the architectural aspects of these quantum computing paradigms to optimize performance and scalability.
    \item Hamiltonian Engineering in Analog Quantum Computing: Developing techniques to design and manipulate Hamiltonians for efficient analog quantum computations.
    
\end{itemize}
The overall goal is to bridge theoretical insights with practical implementations, pushing quantum computing toward tangible and possibly societal advancements.

% \section*{Ongoing projects}

% \subsection*{Project 1: Post-Synthesis Optimization}

% In this project, we focus on optimizing the post-synthesis of quantum circuits on distributed quantum computer. Our approach involves grouping the gates around the two-qubit remote gates that span across two QPUs. We then apply a synthesis algorithm to search for an equivalent circuit that reduces the number of remote gates. This method aims to minimize the overhead associated with remote gate operations, thereby improving the overall performance of quantum circuits on distributed quantum systems.

% \subsection*{Project 2: Multi-Stage Annealing}

% This project introduces a multi-stage annealing process for optimizing quantum circuits on distributed quantum computers. The circuit is divided into different segments or stages, and community detection algorithms are used to generate initial qubit allocations. Subsequently, an annealing algorithm is employed to refine the qubit allocations to minimize the number of remote gates while considering the qubit movement overhead between stages. This novel compilation design takes advantages of the inherent patterns within quantum programs. The multi-stage annealing approach provides a systematic method for compiling large quantum programs, addressing the challenges of qubit allocation and remote gate optimization in DQC.

% \subsection*{Project 3: QIHD}

% This project explores the potentials of quantum inspired algorithms on real-world optimization applications such as vehicle routing. We use GPUs to accelerate the optimization algorithm and achieves advantages over state-of-the-art commercial solvers.


% \section*{RESEARCH EXPERIENCE} 


% \hspace{1.5em}\textbf{Purdue University} \hfill Aug 2022 -- Aug 2024

% ALCHEM Lab,\
% Research Assistant, Advisor: Professor Xuehai Qian

% \textbf{Project: Distributed Quantum Computing and Quantum Pulse Learning}
% % \begin{itemize}
% % \item Reduce the number of remote gates with synthesis technique.
% % \item Develop error mitigation methods for pulse-level quantum computing.
% % \item Evaluate the performance of quantum pulses.
% % \end{itemize}


% \textbf{University of Southern California (USC)} \hfill Aug 2018 -- Aug 2022

% ALCHEM Lab,\
% Research Assistant, Advisor: Professor Xuehai Qian

% \textbf{Project: Quantum Computer Architecture}

% \begin{itemize}
% \item Accelerate quantum optimal control with the similarity graph between groups.
% \item Test generated Qiskit pulse on quantum simulator/device.
% \item Accelerate quantum neural network with natural architecture search.
% \item Optimizing structure of ansatz for QNN/VQE based on device topology.
% \end{itemize}

% \textbf{University of Southern California (USC)} \hfill June 2017 -- May 2018

% ALCHEM Lab,\
% Summer Internship, Advisor: Professor Xuehai Qian

% \textbf{Project: Data-Parallel Finite State Machine}

% % \begin{itemize}
% % \item Test different proposed models with different benchmarks and evaluate the overall performance.
% % \item Optimize parameters of models using performance data.
% % \end{itemize}

% \textbf{Tsinghua University (THU)} \hfill Aug 2017 -- May 2018

% Institute of Microelectronics,\
% Research Assistant, Advisor: Professor Shouyi Yin

% \textbf{Project: Energy-efficient CNN}
% % \begin{itemize}
% % \item Develop tools to evaluate cycle and energy consumption of each layer by layer's parameters.
% % \item Modify networks guided by the cycles and energy consumption in the profiling process.

% % \end{itemize}

% \textbf{Tsinghua University (THU)}, Beijing, China \hfill Aug 2016 -- Feb 2017

% Institute of Microelectronics,\
% Research Assistant, Advisor: Professor Tianling Ren

% \textbf{Project: Vertically Standing Graphene}
% % \begin{itemize}
% % \item Propose applications of vertically standing graphene (VSG) and make experiment plans.
% % \item Grow VSG using microwave plasma-enhanced chemical vapor deposition (MPECVD).
% % \end{itemize}

\section*{Professional Activities}

I serve as a Track Program Committee member for ICCAD 2024 and as a member of the Steering Committee of the Quantum Computer Systems (QuCS) Lecture Series. I also serve as reviewer in journals including QUANTUM and Transactions on Quantum Computing (TQC). 
I received ACM Graduate TA Award from Purdue University Department of Computer Science in 2024.

\section*{Personal}

I'm an amateur astrophotographer with a passion for sports like hiking and badminton. As a teaching assistant, I find great fulfillment in sharing knowledge, answering questions, and educating others about science and technology. My personal website is \url{https://jinleic.github.io/}.





\section*{PUBLICATIONS}


\nocite{*} % cite all bib items, without text

\begingroup % stops printing 'references';
%   http://tex.stackexchange.com/questions/22645/hiding-the-title-of-the-bibliography
\renewcommand{\section}[2]{}%

\bibliography{jinleic}{}
\bibliographystyle{unsrt}

\endgroup

\end{document}